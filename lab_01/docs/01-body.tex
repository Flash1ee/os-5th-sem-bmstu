\section*{Цель работы}
Знакомство со средством дизассемблирования – Sourcer и с получением дизассемблерного кода ядра операционной системы Windows на примере обработчика прерывания INT 8h в virtual mode – специальном режиме защищенного режима, который эмулирует реальный режим работы вычислительной системы на базе процессоров Intel.
\section*{Задание}
Используя Sourser получить дизассемблерный код обработчика аппаратного прерывания от системного таймера INT 8h.

На основе полученного кода составить алгоритм работы обработчика INT 8h. 
\chapter{Дизассемблирование}
\section{Листинг INT8h}
Ниже представлен листинг кода прерывания системного таймера, полученного с помощью программы $sourcer$

\begin{lstlisting}[style = {asm}]
; прерывание системного таймера int8h

; Вызов подпрограммы sub_1.
020A:0746  E8 0070			    call	sub_1; (07B9)

; Сохранение регистров ES, DS, AX, DX в стек.
020A:0749  06					push	es
020A:074A  1E					push	ds
020A:074B  50					push	ax
020A:074C  52					push	dx

; Загрузка в DS 040H, настройка регистров
020A:074D  B8 0040				mov	ax,40h
020A:0750  8E D8				mov	ds,ax
020A:0752  33 C0				xor	ax,ax; Zero register
020A:0754  8E C0				mov	es,ax
; Адрес счетчика прерываний от таймера 0040:006C
; Инкремент младшей части счётчика суточного времени по адресу 0040:006C (2 младших байта)
020A:0756  FF 06 006C			inc	word ptr ds:[6Ch]; (0040:006C=0B995h)
020A:075A  75 04				jnz	loc_1; Jump if not zero

; Инкремент старшей части счётчика суточного времени по адресу 0040:006E (старшие 2 байта)
020A:075C  FF 06 006E			inc	word ptr ds:[6Eh]; (0040:006E=10h)

; Проверка, что прошли сутки. Сброс счётчика реального времени при наступлении новых суток.
020A:0760			loc_1:
020A:0760  83 3E 006E 18		cmp	word ptr ds:[6Eh],18h; (0040:006E=10h)
020A:0765  75 15				jne	loc_2; Jump if not equal
020A:0767  81 3E 006C 00B0		cmp	word ptr ds:[6Ch],0B0h; (0040:006C=0B995h)
020A:076D  75 0D				jne	loc_2; Jump if not equal
; Обнуление двух старших байтов счётчика реального времени (ax = 0)
020A:076F  A3 006E				mov	word ptr ds:[6Eh],ax; (0040:006E=10h)
; Обнуление двух младших байтов счётчика реального времени (ax = 0)
020A:0772  A3 006C				mov	word ptr ds:[6Ch],ax; (0040:006C=0B995h)
; Прошли сутки, установка флага в 0040:0070
020A:0775  C6 06 0070 01		mov	byte ptr ds:[70h],1; (0040:0070=0)
020A:077A  0C 08				or	al,8
020A:077C			loc_2:

; Сохранение регистра AX
020A:077C  50					push	ax

; Декремент счетчика выключения моторчика дисковода по известному адресу в области данных BIOS
020A:077D  FE 0E 0040			dec	byte ptr ds:[40h]; (0040:0040=9Fh)
020A:0781  75 0B				jnz	loc_3; Jump if not zero

; Установка флага, отвечающего за отключение моторчика дисковода
020A:0783  80 26 003F F0		and	byte ptr ds:[3Fh],0F0h; (0040:003F=0)

; Посылается команда 0Ch в порт дисковода 3F2h - отключить моторчик дисковода
020A:0788  B0 0C				mov	al,0Ch
020A:078A  BA 03F2				mov	dx,3F2h
020A:078D  EE					out	dx,al; port 3F2h, dsk0 contrl output

; Вызов прерывания 1 Ch
020A:078E			loc_3:

; Восстановление регистра AX
020A:078E  58					pop	ax
; Проверка, можно ли вызвать маскируемые прерывания (2 бит - Parity flag)
020A:078F  F7 06 0314 0004		test	word ptr ds:[314h],4; (0040:0314=3200h)
020A:0795  75 0C				jnz	loc_4; Jump if not zero
; Косвенный вызов прерывания 1Ch
020A:0797  9F					lahf; Load ah from flags
020A:0798  86 E0				xchg	ah,al
020A:079A  50					push	ax
020A:079B  26: FF 1E 0070		call	dword ptr es:[70h]; (0000:0070=6ADh)
020A:07A0  EB 03				jmp	short loc_5; (07A5)
020A:07A2  90					nop
; Вызов прерывания по таймеру (1Ch)
020A:07A3			loc_4:
020A:07A3  CD 1C				int	1Ch; Timer break (call each 18.2ms)
; Вызов подпрограммы sub_1
020A:07A5			loc_5:
020A:07A5  E8 0011				call	sub_1; (07B9)
; Сброс контроллера прерываний - запись 20h в порт 20h
020A:07A8  B0 20				mov	al,20h
020A:07AA  E6 20				out	20h,al; port 20h, 8259-1 int command
;  al = 20h, end of interrupt
; Восстановление регистров DX, AX, DS, ES
020A:07AC  5A					pop	dx
020A:07AD  58					pop	ax
020A:07AE  1F					pop	ds
020A:07AF  07					pop	es

; Завершение обработчика прерывания 8h
020A:07B0  E9 FE99				jmp	$-164h ; 07B0 - 0164 = 064C -> jmp по адресу 020A:064C

020A:064C			loc_1:
020A:064C  1E					push	ds
020A:064D  50					push	ax
; ...
020A:06AA  58					pop	ax
020A:06AB  1F					pop	ds
020A:06AC  CF					iret; Interrupt return - возврат из прерывания
\end{lstlisting}
\newpage

\section{Листинг подпрограммы sub\_1}
\begin{lstlisting}[style={asm}]
sub_1		proc	near
; Сохранение регистров DS, AX
020A:07B9  1E					push	ds
020A:07BA  50					push	ax
; Установка сегмента данных AX = DS = 0040H
020A:07BB  B8 0040				mov	ax,40h
020A:07BE  8E D8				mov	ds,ax

; Сохранение младшего байта FLAGS в AH
020A:07C0  9F					lahf; Load ah from flags
; Проверка старшего бита IOPL или флага DF
; Если хотя бы один установлен, то IF сбрасывается через cli
020A:07C1  F7 06 0314 2400		test	word ptr ds:[314h],2400h; (0040:0314=3200h)
020A:07C7  75 0C				jnz	loc_7; Jump if not zero
; Сброс IF (9 бит занулить)
; lock - чтобы команда была "неделимой"
020A:07C9  F0> 81 26 0314 FDFF	           lock	and	word ptr ds:[314h],0FDFFh	; (0040:0314=3200h)
020A:07D0			loc_6:
; Загрузка AH в младший байт FLAGS
020A:07D0  9E					sahf; Store ah into flags
020A:07D1  58					pop	ax
020A:07D2  1F					pop	ds
020A:07D3  EB 03				jmp	short loc_8; (07D8)
020A:07D5			loc_7:
; Сброс Interrupt enable flag (IF) с помощью cli
020A:07D5  FA					cli; Disable interrupts
020A:07D6  EB F8				jmp	short loc_6	; (07D0)
020A:07D8			loc_8:
020A:07D8  C3					retn
				sub_1		endp
\end{lstlisting}

\chapter{Схемы алгоритмов}
\section{Обработчик прерываний INT 8h}
\img{150mm}{int8h_1}{Схема обработчика прерываний INT 8h}
\clearpage
\img{180mm}{int8h_2}{Схема обработчика прерываний INT 8h}
\clearpage
\img{180mm}{int8h_3}{Схема обработчика прерываний INT 8h}
\clearpage

\section{Подпрограмма sub\_1}
\img{180mm}{sub_1}{Схема обработчика подпрограммы sub\_1}